\documentclass[a4paper,11pt,DIV=12]{scrartcl}
\pdfoutput=1
\usepackage[dvipsnames,x11names]{xcolor}
\usepackage[ttscale=0.9]{libertine}
\usepackage{mathrsfs}
\usepackage[intlimits]{amsmath}
\usepackage{amssymb}
\usepackage{slashed}
\usepackage{tikz-feynman}
\tikzfeynmanset{compat=1.0.0}
\usepackage{contour}
%\usepackage{feynmp-auto} %feynman diagram
\usepackage{tikz}
\usetikzlibrary{calc,positioning,shadows.blur,decorations.pathreplacing}
\usepackage{booktabs}
\usepackage{float}
\usepackage[titletoc,title]{appendix}
\usepackage[affil-it]{authblk}
\usepackage[numbers,sort&compress]{natbib}
\usepackage{graphicx}
\usepackage[normalem]{ulem}
\usepackage{layouts}
\usepackage{bm}
\usepackage[protrusion=true,expansion,kerning=true,tracking=true,final]{microtype}
\usepackage[%backref,
	        colorlinks=true,
	        linkcolor=hblue,
	        citecolor=hgreen,
	        filecolor=hblue,
	        urlcolor=hred
	        ]{hyperref}
\definecolor{hgreen}{rgb}{0,.3,0}
\definecolor{hred}{rgb}{.3,0,0}
\definecolor{orange}{rgb}{1,0.5,0}
\definecolor{hblue}{rgb}{0,0,.3}
\definecolor{LightGray}{gray}{0.95}
\definecolor{gray}{gray}{0.6}

\newlength{\bibitemsep}\setlength{\bibitemsep}{.035\baselineskip plus .05\baselineskip minus .05\baselineskip}
\newlength{\bibparskip}\setlength{\bibparskip}{0pt}
\let\oldthebibliography\thebibliography
\renewcommand\thebibliography[1]{%
  \oldthebibliography{#1}%
  \setlength{\parskip}{\bibitemsep}%
  \setlength{\itemsep}{\bibparskip}%
}
\usepackage[T1,LY1]{fontenc}

\makeatletter
\DeclareOldFontCommand{\rm}{\normalfont\rmfamily}{\mathrm}
\DeclareOldFontCommand{\sf}{\normalfont\sffamily}{\mathsf}
\DeclareOldFontCommand{\tt}{\normalfont\ttfamily}{\mathtt}
\DeclareOldFontCommand{\bf}{\normalfont\bfseries}{\mathbf}
\DeclareOldFontCommand{\it}{\normalfont\itshape}{\mathit}
\DeclareOldFontCommand{\sl}{\normalfont\slshape}{\@nomath\sl}
\DeclareOldFontCommand{\sc}{\normalfont\scshape}{\@nomath\sc}

\usepackage{scrlayer-scrpage}
\pagestyle{scrheadings}

\allowdisplaybreaks
\setcapindent{1em}

\setkomafont{captionlabel}{\bfseries}
\setkomafont{caption}{\itshape}
\setkomafont{titlehead}{\normalsize\sffamily}
\addtokomafont{title}{\boldmath}
\addtokomafont{section}{\boldmath}

\KOMAoptions{headinclude=false,
             footinclude=false,
             twoside=false,
             parskip=true,
             draft=false,
%             overfullrule=false,
             abstract=true,
             numbers=noenddot,
             DIV=12}

%%%%%%%%%%%%%%%%%%%%%%%%%%%%%%%%%%%%%%%%%%%%%%%%%%%%%%

\newcommand{\Lag}{\mathscr{L}}

\newcommand{\MX}[1]{\textcolor{RoyalBlue3}{\bfseries [MU: #1]}}
\newcommand{\mx}[1]{\textcolor{RoyalBlue3}{#1}}

\newcommand{\ES}[1]{\textcolor{OrangeRed3}{\bfseries [ES: #1]}}
\newcommand{\es}[1]{\textcolor{OrangeRed3}{#1}}

\newcommand{\LDP}[1]{\textcolor{SpringGreen4}{\bfseries [LDP: #1]}}
\newcommand{\ldp}[1]{\textcolor{SpringGreen4}{#1}}

\newcommand{\eds}[1]{\textcolor{Lavender}{#1}}
\newcommand{\EDS}[1]{\textcolor{Lavender}{{\bfseries [EDS: #1]}}}

\newcommand{\Lagr}{\mathcal{L}}
\newcommand{\qed}{$\mathrm{QED}_3$}
\newcommand{\SU}{\mathrm{SU}}

\begin{document}

% TITLE AND AUTHOR
\title{Notes for fermionic 1PI\\ and counter-terms}
\author{Enrico}
\date{\today}  % Automatically updates to today's date

\maketitle

\section{First strategy}

Here follow the calculation for the fermions 1PI at order $O(\frac{1}{k})$ (abelian):
\begin{equation}
    \textcolor{red}{-i\Sigma(\slashed{q})}=-i \int \frac{d^3 p}{(2\pi)^3}\gamma^\mu \frac{2\pi}{k} \epsilon_{\mu \nu \alpha} \frac{p^\alpha}{p^2}\gamma^\nu \frac{(\slashed{p}+\slashed{q})+m}{(p+q)^2-m^2}\ .
\end{equation}
We rewrite the \textbf{denominator} as follows:
\begin{align*}
    \frac{1}{p^2}\frac{1}{(p+q)^2-m^2}&= \int dx \frac{1}{[x((p+q)^2-m^2)+(1-x)p^2]^2}=\int_0^1 dx\frac{1}{[l^2-\Delta]^2}\ ,
\end{align*}
where $l = p +x q\ $ and $\ \Delta = q^2 x(x-1)+x m^2$.

As for the \textbf{numerator}:
\begin{align*}
    -i\frac{2\pi}{k} \gamma^\mu \gamma^\nu \epsilon_{\mu \nu \alpha}p^\alpha (\slashed{p}+\slashed{q}+m)&=-i \frac{2\pi}{k}(g^{\mu \nu}-i\epsilon^{\mu \nu \rho}\gamma_\rho)\epsilon_{\mu \nu \alpha}p^\alpha(\slashed{p}+\slashed{q}+m) 
    \\ &= \frac{-2\pi}{k}(2\delta^\rho_\alpha) \gamma_\rho p^\alpha (\slashed{p}+\slashed{q}+m) =\frac{-4\pi}{k}\slashed{p}(\slashed{p}+\slashed{q}+m)\ .
\end{align*}

Which gives 
\begin{align*}
    \textcolor{red}{-i\Sigma(q)}=\frac{-4\pi}{k}\int_0^1 dx \int \frac{d^3p}{(2\pi)^3} \frac{1}{[l^2-\Delta]^2}(\slashed{p}\slashed{p}+\slashed{p}\slashed{q}+\slashed{p}m)\ .
\end{align*}
Changing to $l$:
\begin{align*}
    \slashed{p}(\slashed{p}+\slashed{q}+m) = (\slashed{l}-x\slashed{q})(\slashed{l}-x\slashed{q}+\slashed{q}+m)=l^2 - x\slashed{l}\slashed{q} + \slashed{l}\slashed{q} + \slashed{l}m -x\slashed{q}\slashed{l}-x\slashed{q}\slashed{q} -x \slashed{q}m +x^2q^2\ .
\end{align*}
Getting rid of linear contributions in $l$
\begin{align*}
    \textcolor{red}{-i\Sigma(q)}=-\frac{4\pi}{k}\int_0^1 dx \int \frac{d^3l}{(2\pi)^3} \frac{l^2}{[l^2-\Delta]^2} - \frac{4\pi}{k}\int_0^1 dx [q^2 x(x-1)-x m \slashed{q}]\int \frac{d^3 l}{(2\pi)^3} \frac{1}{[l^2-\Delta]^2}\ ,
\end{align*}
and Wick rotating:
\begin{align*}
    \textcolor{red}{-i\Sigma(q)}\stackrel{WR}{=} -i\frac{4\pi}{k}\int_0^1 dx \int \frac{d^3l}{(2\pi)^3} \frac{-l^2}{[l^2+\Delta]^2} - i\frac{4\pi}{k}\int_0^1 dx [q^2 x(x-1)-x m \slashed{q}]\int \frac{d^3 l}{(2\pi)^3} \frac{1}{[l^2+\Delta]^2}\ .
\end{align*}
\textbf{NOTE: the first integral clearly shows UV divergences:}
\begin{equation*}
    \int_a^\Lambda d^3 l \frac{l^2}{[l^2+\delta]^2} \sim^{a\to \infty} \int_a^\Lambda dl \frac{l^4}{l^4}\sim \Lambda\ .
\end{equation*}
\textbf{Still, dimensional regulation is blind to power-law divergences, it only shows logarithmic ones.}\\
Let us us dim-reg.\\
Recalling 
\begin{align*}
    \int \frac{d^dl}{(2\pi)^d}\frac{l^2}{[l^2+\Delta]^2}=\frac{d}{2}\frac{\Gamma(1-\frac{d}{2})}{(4\pi)^{d/2}}\frac{1}{\Delta^{1-\frac{d}{2}}}\ \ ;\ \ \int \frac{d^d l}{(2\pi)^3}\frac{1}{[l^2+\Delta]^2}=\frac{\Gamma[2-\frac{d}{2}]}{(4\pi)^{d/2}}\frac{1}{\Delta^{2-\frac{d}{2}}}
\end{align*}
No need to set $d=3-\epsilon$, since the $\Gamma$ functions would show no pole for $\epsilon \to 0$.\\
It gives finally:
\begin{align*}
    \textcolor{red}{-i \Sigma(q)} = &-i \frac{6}{4k}\int_0^1 dx \sqrt{q^2 x(x-1)+x m^2}\\
    &-i\frac{2}{4k}(\int_0^1 dx \frac{q^2 x(x-1)}{\sqrt{q^2 x(x-1)+x m^2}} - \int_0^1 dx \frac{x m \slashed{q}}{\sqrt{q^2 x(x-1)+x m^2}})
    \\ &= -\frac{i}{2k}\int_0^1 dx \frac{1}{\sqrt{q^2 x(x-1)+xm^2}}\left[ 4q^2 x (x-1) + 3xm^2 -xm\slashed{q}  \right]\ .
\end{align*}
From which we obtain the \textbf{$\mathbf{\delta Z_m}$ and $\mathbf{\delta Z_\psi}$ counter-terms} (ON-SHELL scheme) at order $O(\frac{1}{k})$:
\begin{align*}
    \delta Z_m \cdot m = \Sigma(q=m) = \frac{1}{2k}\int_0^1 dx \frac{1}{mx}\left[4x^2m^2-4xm^2+3xm^2-xm^2 \right]=\frac{1}{2k}\int_0^1 dx \left[4xm - 2m \right]=0\ , 
    \\
    \delta Z_\psi=\frac{d}{d \slashed{q}}\Sigma(\slashed{q})\Big|_{\slashed{q}=m}=\frac{1}{2k}\int_0^1 dx \left[ \frac{2m^2x+4m^2(x-1)x}{\sqrt{xm^2+m^2(x-1)x}} \right]=\frac{1}{2k}\int_0^1 dx (2m + 4m(x-1))= 0\ .
\end{align*}

\newpage

\section{Second strategy}

Here follow the calculation for the fermions 1PI at order $O(\frac{1}{k})$
\begin{equation}
    \textcolor{red} {-i \Sigma(\slashed{q})}=-i \int \frac{d^3 p}{(2\pi)^3}\gamma^\mu \frac{2\pi}{k} \epsilon_{\mu \nu \alpha} \frac{p^\alpha}{p^2}\gamma^\nu \frac{(\slashed{p}+\slashed{q})+m}{(p+q)^2-m^2}\ .
\end{equation}

The \textbf{numerator} can be written as:
\begin{align*}
    -i\frac{2\pi}{k} \gamma^\mu \gamma^\nu \epsilon_{\mu \nu \alpha}p^\alpha (\slashed{p}+\slashed{q}+m)&=-i \frac{2\pi}{k}(g^{\mu \nu}-i\epsilon^{\mu \nu \rho}\gamma_\rho)\epsilon_{\mu \nu \alpha}p^\alpha(\slashed{p}+\slashed{q}+m) 
    \\ &= \frac{-2\pi}{k}(2\delta^\rho_\alpha) \gamma_\rho p^\alpha (\slashed{p}+\slashed{q}+m) =\frac{-4\pi}{k}\slashed{p}(\slashed{p}+\slashed{q}+m)
    \\ &=-\frac{4\pi}{k}(p^2 \mathbf{1} + \slashed{p}(\slashed{q}+m))\ .
\end{align*}

Which gives:
\begin{align*}
   \textcolor{red} {-i \Sigma(\slashed{q})} &=\frac{-4\pi}{k} \int \frac{d^3}{(2\pi)^3} \frac{p^2 + \slashed{p}(\slashed{q}+m)}{p^2[(p+q)^2-m^2]}
    \\ &= -\frac{4 \pi}{k}\left[ \int \frac{d^3p}{(2\pi)^3} \frac{\mathbf{1}}{(p+q)^2-m^2} + \gamma_\mu \left( \textcolor{blue}{\int \frac{d^3p}{(2\pi)^3} \frac{p^\mu}{p^2 [(p+q)^2 - m^2]}} \right) (\slashed{q}+m)  \right]\ .
\end{align*}

Focusing on the integral between partenthesis:
in order the preserve the Lorentz structure, and since the only external momentum is $q$, it must be
\begin{align*}
    \textcolor{blue}{\int \frac{d^3p}{(2\pi)^3} \frac{p^\mu}{p^2 [(p+q)^2 - m^2]}} = A(q^2)\cdot q^\mu \ ,
    \\
    A(q^2)= \frac{1}{q^2}\int \frac{d^3p}{(2\pi)^3} \frac{p\cdot q}{p^2 [(p+q)^2 - m^2]}\ .
\end{align*}
Substituting $p\cdot q = \frac{1}{2}[(p+q)^2-m^2 - p^2 - q^2 + m^2]$:
\begin{align*}
    A(q^2) = \frac{1}{2 q^2} \int \frac{d^3}{(2\pi)^3} \left[ \frac{1}{p^2} - \frac{1}{(p+q)^2-m^2} + \frac{m^2 - q^2}{p^2[(p+q)^2 - m^2]} \right]\ .
\end{align*}

Placing it inside the total integral: 
\begin{align*}
    \textcolor{red} {-i \Sigma(\slashed{q})}=-\frac{4\pi}{k}\Biggl\{ \int \frac{d^3p}{(2\pi)^3} \frac{\mathbf{1}}{(p+q)^2-m^2} + \frac{\slashed{q}(\slashed{q}+m)}{2q^2}\int \frac{d^3}{(2\pi)^3} \left[\frac{1}{p^2} - \frac{1}{(p+q)^2-m^2} + \frac{m^2 - q^2}{p^2[(p+q)^2 - m^2]} \right]  \Bigr\}\ .
\end{align*}

Now, usingin dim-reg and recalling
$$
\int \frac{d^d k}{(2\pi)^d} \frac{1}{(k^2 + \Delta)^n}
= \frac{1}{(4\pi)^{d/2}} \frac{\Gamma(n - d/2)}{\Gamma(n)}
\left( \frac{1}{\Delta} \right)^{n - d/2}\ ,
$$
\\
we can evaluate the different integrals inside $\Sigma(q)$. ($WR:$ wick rotation, $DR:$ dim-reg)
\begin{align*}
   a.\  \int \frac{d^3 p}{(2\pi)^3} \frac{1}{(p+q)^2 - m^2}&\ 
= \int \frac{d^3 l}{(2\pi)^3} \frac{1}{l^2 - m^2} \stackrel{WR}{=} -i \int \frac{d^3 l}{(2\pi)^3} \frac{1}{l^2 + m^2} 
\\ &\stackrel{DR}{=}\frac{-i}{8\pi^{3/2}} (-2\sqrt{\pi}) \cdot m = i\frac{m}{4 \pi}
\end{align*}
\begin{align*}
    b.\ \int \frac{d^3 p}{(2 \pi)^3} \frac{1}{p^2} \stackrel{DR}{=} 0 
\end{align*}
\begin{align*}
    c.\ \int \frac{d^3 p}{(2\pi)^3} \frac{1}{p^2 [(p+q)^2-m^2]}&\ =\int_0^1 dx \int \frac{d^3 k }{(2 \pi)^3}\frac{1}{[k^2-\Delta]^2} \stackrel{WR}{=} i \int_0^1 dx \int \frac{d^3 k }{(2 \pi)^3} \frac{1}{[k^2 + \Delta]^2}
    \\ &\stackrel{DR}{=}i\frac{\sqrt{\pi}}{8 \pi \sqrt{\pi}}\int_0^1 dx \frac{1}{(q^2 x(x-1)+xm^2)^{\frac{1}{2}+\varepsilon}}
\end{align*}
Where $l=p+q;\ \ k=p+xq; \ \ \Delta=q^2 x(x-1)+xm^2$.
\\
\textbf{Note:} in the first integral I subsituted $d=3$ directly, whereas in the last one $d=3-2\varepsilon$. The reason is to be found in the counter-term calculation at the end of the page.

This results in: 
\begin{align*}
    \textcolor{red} {-i \Sigma(\slashed{q})}= -\frac{4 \pi}{k}\Biggl\{ i\frac{m}{4\pi} + \frac{1}{2} \left(\mathbf{1}+\frac{\slashed{q}m}{q^2}\right) \left[ -i\frac{m}{4 \pi} + \frac{i}{8 \pi} (m^2 - q^2)\int_0^1 dx \frac{1}{(q^2 x(x-1)+xm^2)^{\frac{1}{2}+\varepsilon}}  \right] \Biggr\}\ .
\end{align*}


Finally, \textbf{the counter-terms}calculation in the ON-SHELL scheme:
\begin{align*}
    \delta Z_m \cdot m = \Sigma(\slashed{p}=m) = 0\ ,
\end{align*}
\begin{align*}
    \delta Z_\psi &= \frac{d}{d \slashed{q}}\Sigma(\slashed{q})|_{\slashed{q}=m} =...=\frac{4 \pi}{k}\Biggl\{ \frac{1}{8\pi}\frac{m^2}{q^2} + \frac{d}{d\slashed{q}}\left[ \frac{1}{8 \pi} (m^2-q^2) \int_0^1 dx \frac{1}{(q^2 x (x-1)+xm^2)^{\frac{1}{2}+\varepsilon}}  \right]  \Biggr\}\Bigg| _{\slashed{q}=m}
    \\ &=\frac{1}{2k} -\frac{1}{k} \int_0^1 dx\  \frac{1}{x^{1+2\varepsilon}}=\frac{1}{2k}\left(1+\frac{1}{\varepsilon}\right)\ .
\end{align*}


% \newpage
% \addcontentsline{toc}{section}{References}
% \bibliographystyle{JHEP}
% \bibliography{references}

\end{document}